\documentclass{article}
\title{
    \begin{center}
            \newline HP VR Training Solution 
            \subsection*{ Fall 2018 - Senior Project}
    \end{center}
}
\usepackage[a4paper,margin=1in,footskip=0.25in]{geometry}
\author{Stephen Hoffmann}
\date{October 2018}

\begin{document}

\maketitle 


\section*{Problem Abstract}
HP trains employees for large printers all over the states by flying them to Corvallis for training. Currently this is a stress on both employees and budget, so the purposed solution is to do virtual training using game technology to streamline training that can only be done at one facility.

\newpage


\section*{Solution Details}
Our team will be using Microsoft's Virtual Reality headset to provide interactive training similar to a real world class. The goal is to create a game that uses the trainee's full body to learn, exactly like a real life program. The trainee will be able to walk in 360 degrees of motion and interact with all objects with in the simulation. This solution's goal is to provide a hands on experience that will be similar to in person training.

\section*{Proposed Solution}
We will be using the Unreal Game Engine to create this solution. Our group will be testing the software on HP's Microsoft Mixed Reality headset(VR) with hand controls. The project will take advantage of 3d assets already existing at HP, as well as their existing training documents, using their Datasmith Application. The project will recreate an environment at HP for printer training. All aspects of training will be handled with simulations based off proper user-action, sound and animation, and in-game responses as close to real life that VR can currently achieve. The headset will track the user's position, height, and rotation, while providing visual and audio feedback. The VR hand-held controllers will track arm motion, finger gestures, and arm motion, while also providing motor-based vibration feedback. We will use sound recorders and audio software to create in-game sound, as well as use cameras to capture factory textures and environmental details for level design. Using all the tools the simulation will attempt to allow the user to make the most out of there sense of Direction, Sound, Sight, and touch. 

Our group will also be meeting HP's current VR Development team for advice, technology, and scope. They will be a tool to help ensure outcomes are met as well as achievable for our client by the end of spring. Along with them we will need to personally go through the training at HP so we know how to incorporate the training into a game. This will involve working closely with our client, HP instructors, and HP's VR Team to ensure requirements and specifications.

The training simulation is in true essence a game, so too enhance the learning environment game mechanics can be used to capture things like high-scores, timers, and puzzles to make the training more dynamic.

\section*{Performance Metrics}
A strong goal for completion will be to have a Game Simulation that trains employees on par to live training. A simple case study can be used to measure effectiveness by having two groups go through one of the training and be tested on the results of the training.
However a conservative goal is important, and I would assume a prototype that covers 90 percent of training documents while still missing some key implementation to the game would still be a success. Other factors would be based on user feedback on game play. This feedback can be done with surveys asking about topics such as intuitive instruction, bugs, visual quality, features, and recommendations. If user responses are typically non-development issues, such as UI tweaks, that would be a good measure of success.


\end{document}
