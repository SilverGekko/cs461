% nick's progress reports:
% week 3:
Progress:

We contacted our client and met with him at block 15 to go over initial information about our VR printing project.

Problems:

We currently don't all have access to VR headsets in order to get familiar with VR games. Our client told of us that he wanted to approach the development of the VR project as a game development process. As "homework" he told us all to go play VR games to get used to how they work and how tutorials work in VR. We all don't have access to the headsets so we need to get some.

Plans:

Get acces to a VR headset and play VR games to do our client "homework".

% week 4:
Progress:

We met as a team to complete the final draft of the project statement.

Problems:

Out client is currently in Israel and cannot meet with us to discuss the requirements document.

Plans:

We will work together to draft our own set of requirements that the customer will review upon returning from Israel.

% week 5:
Progress:

So far our group has been in contact with our sponsor, Tim Holt, about what we should be working on for our group. Tim has said that what we wants us to be doing is playing a few VR games to get ourselves accustomed to the equipment and how VR games are structured vs normal video games. Tim wants to go about making the training application from a game development standpoint and use design philosophies from game design to make a good user experience, specifically taking lessons from tutorials.

Additionally, we also have our outline for our project requirements done and are populating it with our interpretations of what we think the requirements should be.

Problems:

Our project sponsor is still in Israel and cannot devote a lot of time until Monday to helping us with requirements, so much of our requirement document is speculation until our sponsor can get us the training manuals and tell us we are on the right track.

McGrath informed us that the only VR equipment we have access to is in the virtualization lab in Snell hall, which only lets us sign up for hour long blocks. This is fine for us now when all we need to do is play a few games to learn how the tutorials are made and take notes on design. However this poses an issue later in the project if we don't get access to equipment that we can check out for longer periods of time. One hour is not a lot of time to develop games in blocks.

Plans:

After we finish the requirements document we will talk among the group to pick topic that we will write about for our tech reviews since we don't want to accidentally double up on topics.

Also, one of our group member is considering purchasing their own VR headset for personal use. This has the added benefit that we would be able to use it for our gorup development and not be time restricted. Since two other group member already own VR headsets of some kind, this would speed up development by a lot if we didn't have to use the Snell hall resources to develop.
%week 6:
Progress:

Our group was finally able to meet in person with our project sponsor, Tom Holt, at HP to take a tour of the Web Press printers and talk more about expectations of the project. Our group also drafted a requirements document, but it turned into too much of a design document so we created a brand-new document that better fit the "requirements" document description: describing what a solution to the problem statement is required to do, instead of how to do it, which is what our document turned into.

Problems:

Our project sponsor is requiring us to sign an NDA. However, this NDA is not for the entire project, it's just for HP confidential assets such as 3D models and training manuals.

Plans:

Even though we are being required to sign an NDA to use HP assets, it will not get in the way of presenting our projet. We will be allowed to create derivative works from the HP-provided CAD files that will not be HP propriety-owned.
%week 7:
Progress:

There was not a ton of work for us to do this week. We met in class on Tuesday to peer-review our tech reviews and I got some good feedback about the flow of my paper. I ended up not having to change much from the first draft, since I completed most of it for the first assignment.

Problems:

Our resident game design expect, Stephen, says that we might want to start installing and going through tutorials in the Unreal Engine as it has a steep learning curve. I foresee this being especially challenging for myself as I have very little professional game development experience to my name.

Plans:

Over the remaining weeks of the term I plan on watching Unreal Engine tutorial videos to get up to speed with the rest of my group before the work load starts in Winter term. 

Our team will also work as a group to complete our design document draft, making sure to include the specific user stories that our client request. Last week our client told is that these users stories should include the proper sound ambiance as what is found in the actual R&D location at HP.
%week 8:
Progress:

This week there was not a lot of things for us to do. Last Friday we finished our Tech Reviews and this week in class we talked all about design documents, and got lectured at by McGrath about using spellcheck.

Problems:

Our client has not been very attentive in answering emails. We sent him a progress update for our documentation last Friday and we still have not received word back from him. I also do not feel inclined to come to Thursday lecture if I have to listen to McGrath belittle me as a student for having poor writing.

Plans:

For the coming holiday weeks we are going to work on the design document to finish it up. We got information from Chris that said each section of the document needs to be 500 words or longer, to ensure that we only write about things that have content.

Also for next term during the design phase we are going to try and set up a weekly meeting with our client so that both us and him feel like we have a good understanding of the project moving forward.
%week 9:
Progress:

We have not made any progress this week on our project.  We did receive an email from our project sponsor Tim about the state of the NDA we have to sign to get access to the source files. We sign it on our own and bring it to the HR person at HP that oversees Tin's department. Again: this NDA only coves the base .stl CAD files that we will be using as a reference, and does not extend to our project in any way. The final project will not be secret and we will be able to call it our own and present it at the career fair.

Problems:

My tech review has not yet been graded by McGrath or Kirsten so I am unable to review feedback about my writing to apply it to my sections of the design document.

Plans:

This weekend we will finish the design document making sure to follow the ISO standard properly this time. I will ask another one of my group members about their feedback for their tech review in order to better prepare myself for my sections of the design document.