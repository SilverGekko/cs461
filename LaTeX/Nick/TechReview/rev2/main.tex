\documentclass[onecolumn, draftclsnofoot,10pt, compsoc]{IEEEtran}
\usepackage[utf8]{inputenc}
\usepackage{csquotes}
\usepackage[hyphens,spaces,obeyspaces]{url}
\usepackage{hyperref}
\title{Tech Review: Video Game Tutorials vs. Traditional Teaching Styles}
\author{Nicholas Pugliese}
\date{December 2018}
\newpage
\begin{document}
\maketitle

\section{Introduction}
Our project is to create a virtual reality simulation in order to teach the user how to operate and maintain the HP Web Press from a remote location, augmenting the existing in-person seminars that HP holds for customers. The training program will share many of the same characteristics as a video game, and our team will utilize video game design principles to teach the user how to use the Web Press. This shift in paradigm changes how information will be passed to the user and how the user will be tested for knowledge retention. This tech review will compare and contrast "classroom" environments such as a traditional lecture/seminar style of teaching versus that of an interactive video game/program. The "technologies" being reviewed are methods of education, which can be classified as technology based on the first definition of technology from Merriam-Webster.

\begin{displayquote}
"\textbf{technology}: the practical application of knowledge especially in a particular area"
\end{displayquote}

\section{Traditional}
The most common style of in-person teaching is the lecture format. Lectures involve a speaker or speakers talking to a large number of students about a topic or topics. While lecture formats are usually enhanced by means of visual aid in the form of either physical objects or a slideshow of images, they are generally oral communication. Students in lectures take notes and memorize lessons to retain information \cite{lecture}. An in-person lecture has time set aside during class for questions and clarifications. Achieving the same student/teacher interaction can be difficult in a remote environment. Prerecorded lectures also exist, and those may serve distance learning students better because they have the ability to pause and go back to rewatch a section over to let the information sink in and take better notes. However, this method lacks the near-instant feedback a student can receive by asking a question directly to the lecturer.

Lecture formats are really good at teaching ideas and concepts. Learning how to operate a machine requires hands-on experience. A format for teaching this kind of hands-on knowledge, is often a laboratory or "Lab" class. Lab classes are usually longer than lectures (usually lasting 2-3 hours) and involve completing a task or tasks and documenting what happened in a formal write up. Lab work is usually a more-involved task that acts like a piece of homework. Another benefit of a lab format is that the lab teacher is present and can be asked questions.

Lecture based formats allow experts to share their knowledge directly with the students. It is a great method for hammering in information in sequence that builds upon itself and can include showing example equations as well as work that can be done in class with pen and paper. This style would lend well to general information about the Web Press. However, the ceiling for useful skill knowledge can only reach so high. If a lecturer is describing how to do an action the student can write it down and memorize it, but there are fundamental lessons (such as learning to operate machinery) that can only be taught through through experience.

\section{Video Games}
Video games have the ability to implicitly teach the player with actions rather than words. For example, the video game Portal by Valve revolves around the namesake device the "portal gun." The gun is capable of firing two portals, one orange, one blue, on a surface. If you pass through one portal you appear out the other. The tutorial of Portal demonstrates this ability to the player by placing the device on a rotating pedestal in the center of a room. The player is forced to wait inside of an anti-chamber to the room the Portal Gun is in, connected by a locked door and a window. Inside the room the orange portal is stationary on a wall the player can clearly see and as the portal gun revolves it fires the blue portal onto different walls in the room. The player is encouraged to watch what the portal gun is doing simply by having nothing else to do in the room. Because of this funneling of the player's attention and subtle display of the game's main mechanic, the player has learned what they will need to do for the majority of the game without having to resort to a voice over or an essay of text giving instructions. The player has been taught without breaking immersion. \cite{portal}

Half Life 2 by Valve uses this experimental style of teaching to pass knowledge to the player. In the beginning tutorial of Half-Life 2 the player is put into a room with several wooden creates and a wooden barricade placed over the exit door. Placed in the center of the room there is a metal crowbar. Through experimentation the player must realize that they can smash the wooden crates and eventually the obstruction over the exit door. This implementation of the experimental tutorial is different from the one in portal that is more guided. In Half-Life 2 the player may sit in the room for as long as they wish. The game will not prompt them to continue. It is solely player-diver to proceed in the game. This situation presented to the player early in the game sets the tone that things may be played around or experimented with, and that behavior will be rewarded.
\cite{hl2}

\section{Comparison}
Video games have the ability to create situations like in Half-Life where it is up to the player/student to figure out how to proceed. This engagement requires the student to use their critical thinking skills rather than memorize information from a book or lecture. As the old proverb says" "If you give a person a fish, they will eat for a day. If you teach a person to fish, they'll eat for a lifetime." The essence of this is that learning how to do something lays the foundation of understanding so you can replicated the actions in the future. This is why creating a virtual reality training program will be vastly better than just recording a bunch of videos for the customers to watch. If they customers are engaged in their learning and have to figure things out they will not only retain more knowledge but also be able to apply the lessons learned to different situations in the future.

Nick Babich in an article he wrote for Adobe says:
\begin{displayquote}
"Too much information received in a short period of time can easily overwhelm students. As a result, they become bored, disengaged, and usually not sure why they are learning about a topic in the first place." \cite{adobe}
\end{displayquote}
Granted, there are some students who will thrive in a fact-memorization environment. Part of the recommendation of this paper is to create two modes for our project to account for these different learning patterns from different kinds of people. The main area that can be explored in VR that can't be taught in lecture is this sense of experience. Babich also writes:
\begin{displayquote}
"When students read about something, they often want to experience it. With VR, they aren't limited to word descriptions or book illustrations; they can explore the topic and see how things are put together." \cite{adobe}
\end{displayquote}
With this added tool of visual learning students that might have been struggling to understand a concept can immediately see the merit of what they are being told to do. Results can formulate right in front of their eyes that reinforces their understanding of the material.

Video games can embody this experimentation style of learning because they can create invisible barriers or teaching devices (such as the door unlocking only after you watch the portal gun for a while) that still act as a teacher, but are player reliant. This style of teaching by experimentation won't appeal to all students or players, but for the ones it does it creates a sense of accomplishment and internal motivation that bolsters the student/player's desire to play or learn more.

This style of putting the player in a metaphorical box is one that is not unique to new media such as video games and virtual reality, it has existed in the form of puzzles for a very long time. Teaching via puzzles like the ones in video games uses the same techniques as a laboratory class: forcing the student to solve a problem using a limited set of information. Video games intrinsically have to employ this technique because game (and virtual reality programs) are interactive experiences. Not all games use the Portal and Half-Life 2 style of tutorial. Some games present the user with text descriptions of actions and the user must read to find out what things do. This tactic of bringing the user outside of the game to teach is not considered to be effective. \cite{tutorials}

\section{Recommendation}
For our project, I recommend to our group that we focus on creating a learning environment for two types of learners: those who learn by reading or from lectures ("readers"), and those who learn by doing ("doers"). This division of learners is made because virtual reality allows for the creation of an almost one-to-one ratio of real life hardware to virtual reality simulation. This simulation provides a large boost to the retained knowledge of "doers" because they are allowed to learn by experimentation and get to see a firsthand account of the knowledge in action. For "doers" a training mode will be created using the style of tutorials from Portal and Half-Life 2. This mode is analogous to a sandbox learning environment where the user has a goal, but no set of rules or guidelines to achieve that goal. They are free to do as they wish as long as the overarching task is completed. Our group will be examining more video games such as Portal and Half-Life 2 to find more example of good tutorials by experimentation to base the sandbox training scenarios off of. To keep the learner on track, a series of checkboxes or progress meters will be displayed on the heads-up-display of the user-interface. These progress meters don't specify how to complete their goals and instead only track if they have been completed. This visual indicator of program will help the "doers" to stay motivated in experimenting with the sandbox.

The second learning environment I recommend our group to make is one for "readers." Learning by fact retention still remains a valid method of teaching, as evidenced by the hundreds of thousands of school using this method all over the world. If the sandbox mode just doesn't cut it for a learner, there needs to be an alternative. This alternative is a module-based lecture format that still requires the learner to go through the motions themselves to complete the task, but it preceds any user action with visual instructions on the screen overlaid on top of pictures of what the user is supposed to accomplish. This module mode will work better for learners that like to read up on what they are doing before jumping into the action. In addition to text on the screen, a voice over is an option the learner can opt into to go long with the text. Like the sandbox, the module mode will have progress meters on the screen available for the leaner to reference, with the difference that the user can click on or touch the progress meter to reactivate the module's instructions if they need to hear them again. The available repetition of the instructions will help the "reader" to not get lost in the module without direction.




\begin{thebibliography}{00}
\bibitem{lecture} 'Teaching with Lectures', [Online]. Available: \url{https://teachingcenter.wustl.edu/resources/teaching-methods/lectures/teaching-with-lectures/} [Accessed: 31- Oct- 2018]

\bibitem{portal} Portal, Oct 10, 2007, [Video Game]. Available: \url{https://store.steampowered.com/app/400/Portal/} [Accessed: 31- Oct- 2018]

\bibitem{hl2} Half-Life 2, Nov 16, 2004, [Video Game]. Available: \url{https://store.steampowered.com/app/220/HalfLife\_2/} [Accessed: 31- Oct- 2018]

\bibitem{adobe} N. Babich 'How Virtual Reality Will Change How We Learn and How We Teach'. Jan. 9, 2018. [Online]. Available: \url{https://theblog.adobe.com/virtual-reality-will-change-learn-teach/} [Accessed: 3- Dec- 2018]

\bibitem{tutorials} P. Suddaby,'The Many Ways to Show the Player How It's Done With In-Game Tutorials', August 31, 2012, [Online]. Available: \url{https://gamedevelopment.tutsplus.com/tutorials/the-many-ways-to-show-the-player-how-its-done-with-in-game-tutorials--gamedev-400} [Accessed: 1- Nov- 2018]


\end{thebibliography}
\end{document}
