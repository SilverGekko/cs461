\documentclass[onecolumn, draftclsnofoot,10pt, compsoc]{IEEEtran}
\usepackage[utf8]{inputenc}
\usepackage{hyperref}
\title{Tech Review: Video Game Tutorials vs. Traditional Teaching Styles}
\author{Nicholas Pugliese}
\date{October 2018}
\newpage
\begin{document}
\setlength\parindent{0pt}
\maketitle

\section*{Introduction}
Our project (creating a virtual reality training program for HP industrial "Web Press" printers) is to create a virtual reality simulation in order to teach the user how to use the HP Web Press from a remote location. The training program will share many of the same characteristics as a video game, and as such our team will utilize video game design principles to teach the user and how to use the Web Press. This shift in paradigm will pose changes to how information will be passed to the user and how the user will be tested for knowledge retention. This tech review will go into detail about the difference in the "classroom" environments such as a traditional lecture/seminar style of teaching versus that of an interactive video game/program.

\section*{Traditional}
The most common style of in-person teaching is the lecture format. Lectures involve a speaker or speakers talking to a usually large number of students about a topic or topics. While lecture formats are usually enhanced by means of visual aid in the form of either physical objects or a slideshow of images, they are generalized as of oral communication. Students in lectures usually take notes and memorize lessons to retain information \cite{lecture}. An in-person lecture usually has time set aside during class for questions and clarifications. This can be hard to achieve in an remote environment because the in-person instructor can moderate questions by calling on students with raised hands and a live lecture over a video chat system will not have that feature. Prerecorded lectures also exist, and those may serve distance learning students better because they have the ability to pause and go back to rewatch a section over to let the information sink in and take better notes. However, this method lacks the near-instant feedback a student can receive by asking a question directly to the lecturer.

Lecture formats are really good at teaching information that doesn't require the students to actually work on a machine or tool like the Web Press. You can talk about its operation all you want but the best way to gain experience in things is actually working on the machine or tool.

A format of teaching these hands-on knowledge, usually called a laboratory or "Lab" class. Lab classes are usually longer than lectures (usually lasting 2-3 hours) and involve completing a task or tasks and documenting what happened in a formal write up. Lab work is usually a more-involved task that acts like a piece of homework that is usually completed in a group. A benefit of doing this in a lab format is that the lab teacher is present and can be asked questions.

Lecture based formats allows for experts to share their knowledge with the students. It is a great format for hammering in information in sequence that can either build upon itself or show examples equations and work that can be done in class with pen and paper. This style would lend well to general information about the Web Press. However, the ceiling for useful skill knowledge can only reach so high. If a lecturer is describing how to do an action the student can write it down and memorize it, but there are fundamental lessons that can only be learned through doing. 

A combined lecture/laboratory setting is ideal for teaching theory information and applying theory and physical skills in a controlled setting that allows for students to get real experience.

\section*{Video Games}
Video games have the ability to implicitly teach the player with actions rather than words. The video game Portal by Valve revolves around the namesake device the "portal gun." The gun is capable of firing two portals, one orange, one blue, on a surface. If you pass through one portal you appear out the other. The tutorial of Portal demonstrates this ability to the player by placing the device on a rotating pedestal in the center of a room. The player is forced to wait inside of an anti-chamber to the room the Portal Gun is in, connected by a locked door and a window. Inside the room the orange portal is stationary on a wall the player can clearly see and as the portal gun revolves it fires the blue portal onto different walls in the room. The player is encouraged to watch what the portal gun is doing simply by having nothing else to do in the room. Because of this funneling of the player's attention and subtle display of the game's main mechanic, the player has learned what they will need to do for the majority of the game without having to resort to a voice over or an essay of text giving instructions. The player has been taught without breaking immersion. \cite{portal}

Half Life 2 by Valve uses this experimental style of teaching to pass knowledge to the player. In the beginning tutorial of Half-Life 2 the player is put into a room with several wooden creates and a wooden barricade placed over the exit door. Placed in the center of the room there is a metal crowbar. Through experimentation the player must realize that they can smash the wooden crates and eventually the obstruction over the exit door. This implementation of the experimental tutorial is different from the one in portal that is more guided. In Half-Life 2 the player may sit in the room for as long as they wish. The game will not prompt them to continue. It is solely player-diver to proceed in the game. This situation presented to the player early in the game sets the tone that things may be played around or experimented with, and that behavior will be rewarded.
\cite{hl2}

\section*{Comparison}
Video games have the ability to create situations like in Half-Life where it is up to the player/student to figure out how to proceed. This engagement requires the student to use their critical thinking skills rather than memorize information from a book or lecture. As the old proverb says" "If you give a person a fish, they will eat for a day. If you teach a person to fish, they'll eat for a lifetime." The essence of this is that learning how to do something lays the foundation of understanding so you can replicated the actions in the future. This is why creating a virtual reality training program will be vastly better than just recording a bunch of videos for the customers to watch. If they customers are engaged in their learning and have to figure things out they will not only retain more knowledge but also be able to apply the lessons learned to different situations in the future.

Video games can do this because you can create invisible barriers or teaching devices (such as the door unlocking only after you watch the portal gun for a while) that still act as a teacher, but are player reliant. This style of teaching by experimentation won't appeal to all students or players, but for the ones it does it creates a sense of self accomplishment that bolsters the student/player's desire to play or learn more.

This style of putting the player in a metaphorical box is one that is not unique to new media such as video games and virtual reality, it has existed in the form of puzzles for a very long time. Teaching via puzzles like the ones in video games uses the same techniques as a laboratory class: forcing the student to solve a problem using a limited set of information. Video games intrinsically have to employ this technique because game (and virtual reality programs) are interactive experiences. Not all games use the Portal and Half-Life 2 style of tutorial. Some games present the user with text descriptions of actions and the user must read to find out what things do. This tactic of bringing the user outside of the game to teach is not considered to be effective. \cite{tutorials}

\section*{Reccomendation}
Lecture formats are great ways of teaching; it's the worlds leading education style for a reason. It can lead to a lot of understanding of the theoretical. Homework and labs help to solidify skills. However, there are some skills that need to be learned on the actual machine or tool that can't be done in a lecture or homework. This need to practice is showcased in video games: you can play the tutorial only so many times before you need to start level one to put your knowledge of theory to the test. Both teaching styles have their merit, and are suited for different situations.

For our project I recommend that we select the video game style of teaching. I believe that the customers to be trained on the Web Press will gain valuable experience be being forced to learn by doing, rather than being told what to do.


\begin{thebibliography}{00}
\bibitem{lecture} 'Teaching with Lectures', [Online]. Available: \url{https://teachingcenter.wustl.edu/resources/teaching-methods/lectures/teaching-with-lectures/} [Accessed: 31- Oct- 2018]
\bibitem{portal} Portal, Oct 10, 2007, [Video Game]. Available: \url{https://store.steampowered.com/app/400/Portal/} [Accessed: 31- Oct- 2018]
\bibitem{hl2} Half-Life 2, Nov 16, 2004, [Video Game]. Available: \url{https://store.steampowered.com/app/220/HalfLife\_2/} [Accessed: 31- Oct- 2018]
\bibitem{tutorials} P. Suddaby,'The Many Ways to Show the Player How It's Done With In-Game Tutorials', August 31, 2012, [Online]. Available: \url{https://gamedevelopment.tutsplus.com/tutorials/the-many-ways-to-show-the-player-how-its-done-with-in-game-tutorials--gamedev-400} [Accessed: 1- Nov- 2018]


\end{thebibliography}
\end{document}
