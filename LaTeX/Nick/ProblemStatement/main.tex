\documentclass[draftclsnofoot,onecolumn]{IEEEtran}
\IEEEoverridecommandlockouts
\usepackage{cite}
\usepackage{amsmath,amssymb,amsfonts}
\usepackage{algorithmic}
\usepackage{graphicx}
\usepackage{textcomp}
\usepackage[left=2cm, right=2cm, top=2cm]{geometry}
\usepackage{xcolor}
\usepackage{hyperref}
\def\BibTeX{{\rm B\kern-.05em{\sc i\kern-.025em b}\kern-.08em
    T\kern-.1667em\lower.7ex\hbox{E}\kern-.125emX}}
\begin{document}
\title{Virtual Reality Training Program for HP WebPress Printing Technology}
\author{Nicholas Pugliese\\
CS462: Senior Capstone Software Design, Fall 2018\\
}
\maketitle


\begin{abstract}
This document is a project statement for a virtual reality training program for an HP Inc Page Wide Industrial Web Press printer. HP Wants to train customers that wish to purchase the industrial web press printer without having to spend a lot of money to fly the customers out to the printer on site for training. A virtual reality program will cut these costs of training customers significantly, with the added benefit of not having to shutdown production of one of HP's Web Presses for the duration of the training. The virtual reality program created by our team will contain multiple training scenarios including situations where the Web Press printer will break, as well as an interactive mode where the user will be able to explore the web press at their own pace. If the scope of the project allows for it, we will create multiple training program, one each for the different models of web press sold by HP.
\end{abstract}
\newpage
\section{Background}

HP's main business is printers. One of the products they sell is called the "HP Page Wide Industrial Web Press." Is it a room-sized printer capable of printing up to 600 feet per minute. They are used to print anything from poster to detailed product boxes (such as kitchen aid mixers) to junk mail runs. The printers are efficient at printing large batch jobs because the print heads are stationary and the paper is transferred from a spool onto the conveyor belt system that rolls the paper past the print heads at great speed. The print head bar is exactly as wide as the paper for maximum efficiency. One application of these page wide web presses is printing books for Amazon. If supported, when you purchase a book from Amazon's website a job is sent to the print queue for your book to be printed. The HP page wide web press then prints all the pages of your book and then sends it to a binding station to be bound before it is sent to you. This design saves Amazon space in their warehouse because they do not have to keep large amounts of books in storage waiting to be bought.

Additionally, if the current web press is not capable of doing exactly what a customer needs an industrial printer to do, HP will work with the customer to provide a custom web press that will be suited for the type of jobs the customer needs to have printed. HP offers several models of web press suited for a variety of needs. After the customers acquire a web press HP continues to provide hardware and software support, usually in the form of user interface design updates that include new, customer requested features.

\section{Problem Description}
When a company wants to purchase a Web Press, they must be trained on how to operate it and how to troubleshoot it. This can cost HP and said company a lot of time and money because representatives of the company must be flown out to HP's Corvallis campus and put up in hotel rooms for the duration of the training. Additionally, all of the Web Presses at HP are constantly in use. The machine's current job schedule would have to be interrupted to make time for the company's representatives to go through the training program. Additionally, certain scenarios could not be run through such as when the machine breaks down or another catastrophic failure occurs.

\section{Proposed Solution}
All these logistic and monetary problem could be solved if HP could bring a Web Press to the customers. With virtual reality, this is possible. If HP had a virtual reality simulation of the Web Press printer all they would have to do to train potential and current customers would be to ship them a virtual reality headset and a copy of the training program and they could be trained without all the cost of the on-site experience. Even more cost could be saved if the company already owned or had access to a virtual reality headset. The training program could also be utilized as a piece of advertising. Potential customers could be granted access to the training program  (or a special demo application bundled built separately) that would give them a better understanding and knowledge about how the web press works without having to be on-site with a real web press at HP or at another customer's web press location.

\section{Performance Metrics}
The goal of the virtual reality training program is to train representatives from a company on how to use the "HP Page Wide Industrial Web Press" with the same accuracy as if the training were conducted in person. The virtual reality training program must be as fully featured as the real web press in order to provide the most accurate training. Additionally, the training program must provide simulations of when catastrophic events cause the web press to fail. This failure mode training should contain a random test mode where a problem in the web press occurs and the trainee must figure out the solution to the problem.

The demonstration will have two modes: interactive and non-interactive. The non-interactive mode would play out similarly to a movie. The user would be able to move around in the virtual world while a series of predefined, scripted events would transpire for the user to watch. The interactive mode would function as a playground for the user. The web press would sit idle waiting for the user to interact with its user interface. In this mode the user would have access to unlimited resources for the web press such as ink barrels, paper rolls, and a variety of other printing mediums such as cardboard boxes and junk mail.

The scope of this project is to create a virtual reality training program for just one industrial web press. However, if it is within our timeline, we will create multiple program for the different web presses HP offers to provide a large variety in available training that current and potential customers can choose from.

\end{document}