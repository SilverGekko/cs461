TEAM NUMBER: 73 - Virtual Reality Printing 

PROJECT NAME: Developing Virtual Reality Based Training Experiences For HP's PageWide Web Press Product Line

HIGH LEVEL GOAL: Improve the quality of learning and knowledge of HP's PageWide Web Press equipment by implementing virtual reality based training scenarios.

ROLES: 
    - Symon
        - Virtual Reality vs Augmented Reality vs Mixed Reality
        
        

Introduction
    Define what VR/AR/MR are
    Purpose of VR/AR/MR
    Introduce Differences
Virtual Reality
    When to use 
Augmented Reality 
    When to use
Mixed Reality
    When to use
Conclusion


------------
https://www.ijcsmc.com/docs/papers/June2017/V6I6201748.pdf

Mehroosh Sidiq et al, International Journal of Computer Science and Mobile Computing, Vol.6 Issue.6, June- 2017, pg. 324-327

Augmented Reality VS Virtual Reality

Augmented Reality (AR) and Virtual Reality (VR) are a part of these advanced and innovative forms
of technologies that were thought as a part of fiction even a few years ago.

Both Augmented Reality and Virtual Reality have the same goal of immersing the user into a virtual world. With
AR, users continue to be in contact with the real world while interacting with the virtual objects around them,
whereas with VR the user is far away from the real world while completely immersed into the virtual world.


In Augmented Reality, users participate in the physical environment and with other users directly along with
computer simulated virtual objects embedded in the environment. In Virtual Reality, users participate in the visual
environment that is completely mediated

In Virtual Reality, the user fails to sense or accept the existence of his surroundings and
reacts as if the environment is not there, whereas in Augmented Reality, the user can sense and accept the existence
of his surroundings and accordingly reacts to his environment

Augmented Reality and Virtual Reality allow experience that are evolving more commonly than expected and are
attaining high standards in various fields like entertainment, science, medicine, visualization and annotation, robot
path planning, military aircraft, etc.
Augmented Reality is ahead of Virtual Reality, as there are several products already in the market. Virtual Reality
has its limitations. 


--------------
Mana Farshid, Jeannette Paschen, Theresa Eriksson, Jan Kietzmann,
Go boldly!: Explore augmented reality (AR), virtual reality (VR), and mixed reality (MR) for business,
Business Horizons,
Volume 61, Issue 5,
2018,
Pages 657-663,
ISSN 0007-6813,
https://doi.org/10.1016/j.bushor.2018.05.009.
(http://www.sciencedirect.com/science/article/pii/S000768131830079X)
Keywords: Augmented reality; Virtual reality; New technologies; Real constructs; Possible constructs; Information shadow


https://www-sciencedirect-com.ezproxy.proxy.library.oregonstate.edu/science/article/pii/S000768131830079X

Go boldly!: Explore augmented reality (AR), virtual reality (VR), and mixed reality (MR) for business


The actual reality continuum deals with elements that exist or experiences that take place in the actual, material world around us—either in a pure sense, without the help of any type of IT, or with additional data about the actual world provided to us through IT. This continuum includes reality in its authentic and genuine sense as well as augmented reality.

Augmented reality (AR) refers to the integration of the actual world with digital information about it. Actual objects and people cast an information shadow: an aura of data which, when captured and processed intelligently, can offer extraordinary value to consumers (O’Reilly & Battelle, 2009). Augmented reality uses technology to make such a layer of information accessible to people—to blend one’s perception of the actual world with digital content about it generated by computer software. This technology comes in a myriad of forms: from wearables and smart glasses that use retinal projection to put a display in the wearer’s eyeball (e.g., Google Glass was a very noticeable AR headset, the Vaunt by Intel is much less conspicuous) to the more commonly used smartphones. The AR layers that are added can be sensory (e.g., sound, video, graphics, or haptics) or simply data based. To follow the real estate example above, Realtor.com is developing an AR application that allows house hunters to explore a neighborhood in creative new ways. When a phone’s camera points at a home, even if it is not for sale, the app instantly displays information about it, including the last sale prices, taxes, and lot size. This allows the user to judge whether an actual listing is a fair market price.



In terms of business opportunities, AR applications can be used to change how today’s always-connected consumers work and shop. By providing additional information about actual offerings, AR can enhance a customer’s real-world experience in interesting ways. For instance, using an app called 19 Crimes, consumers can point their smartphone at a bottle of 19 Crimes wine and watch the convicts depicted on the actual wine labels come to life (Cawley, 2017). With AR-enabled smart glasses or phones, shoppers could easily walk down an aisle and identify groceries on shelves that fit their dietary restrictions (e.g., those that are gluten-free, non-GMO, nut-free). Similarly, AR could lead a shopper through the store to find items on a shopping list, make recommendations for complementary products, and keep a running total of items in the cart. Using GPS mobile apps with AR allows businesses in tourism to show visitors routes and directions to desirable destinations and to provide additional information for cultural events or historical sites. Another way of adding data on top of the actual world includes using AR applications such as Google Translate for real-time voice and text translation in customer service encounters or in international meetings (Russell, 2015).



In contrast to the actual reality continuum, we have chosen the term virtual reality continuum to reflect all the different types of realities that people commonly associate with virtual reality. As we show below, virtual reality is actually just one type of reality—the beginning of the virtual reality continuum—which also includes mixed reality, augmented virtuality, and virtuality. Each type of reality on the virtual reality continuum is completely virtual; none include any actual, physical items at all, just various combinations of digital versions of what already exists in actuality and what is possible.


.1. Virtual reality
Virtual reality (VR) refers to complete, 3-D virtual representations of the actual world or of objects within it. For instance, AutoCAD software allows architects, engineers, and design professionals to create precise 3-D drawings of actual buildings before they make changes to them. Virtual 360-degree tours invite others to visit faraway sites. An example of this is the website of the Metropolitan Museum of Art (the Met) in New York, which invites visitors to a virtual tour of the works of van Gogh and di Bondone. The most impressive virtual realities, though, are scenarios that require VR headsets that look like giant matte black ski goggles. When users wear an Oculus Rift or Samsung Gear VR, they become completely immersed in 3-D computer-generated worlds. Fully realistic images, sounds, and other sensations simulate a user’s physical presence of environments that actually exist. Some prominent examples of VR include:
•
Google Art and Culture recently launched VR tours of more than 1,200 museums and exhibitions of art;

•
President Obama and his wife, Michelle, offer their expertise as VR tour guides of the White House; and

•
People for the Ethical Treatment of Animals (PETA) created a VR movie called I, Chicken so that viewers can experience the actual conditions of mass animal husbandry.


There are plenty of business applications for virtual reality. Education and training is just one field in which VR seems to offer nearly unlimited business opportunities. Unlike an actual world training scenario, an employee can play through a VR simulation as many times as required to grasp a concept, task, or procedure—regardless of whether it is sales or public speaking (e.g., oculus virtual speech), medical surgery, or operating heavy-duty equipment. In education, for example, virtual reality technology makes learning more engaging by allowing students to interact with virtual models in subjects such as medicine, cosmology, physics, geography, and biology, to name a few. History students can tour an exact virtual replica of the Roman Colosseum, experiencing the monument firsthand without having to travel significant distances (Petch, 2016). Thanks to VR, businesses can reach out to their customers via immersive and engaging marketing campaigns. This is particularly important in the era of online shopping, as VR experiences help customers explore 3-D renderings of a firm’s offering without having to leave their homes.


Virtual reality is different from reality and augmented reality in a very important way, which also sets the foundation for all remaining types of realities on the virtual reality continuum. Unlike a traditional computer, mobile phone, smart spectacles, or a TV set in which users can look to the left and to right and realize that they are still in the actual world (e.g., their living rooms), virtual reality surrounds users as if they were looking or moving in a reality that is different from their actual reality. One can remain in the living room and virtually experience what it’s like to go skydiving, visit famous places, or fly through the Arctic. The difference is that VR can be fully immersive. Users often forget where they actually are and even experience a phenomenon called VR sickness: the purely visually induced perception of self-motion (rather than actual motion) that leads to symptoms like disorientation, discomfort, headache, and nausea. Such actual symptoms—resulting from perceived presence—show how powerful VR can be and how important it is that care is taken not to misuse its powers.



 Mixed reality (MR) refers to the merging of real world virtual constructs with computer-generated constructs that are either real or possible. Not only does mixed reality (also known as hybrid reality) combine aspects of the actual reality—the physical world around us—with the power of virtual reality, it also combines what’s real with what’s possible. In other words, mixed realities allow us to experience new objects or scenarios—those that don’t actually exist. Imagine adding virtual objects or characters into a live video stream of the real world. In healthcare, for example, medical mannequins are brought to life for training scenarios and teach empathy to healthcare professionals (DeSouza, n.d.). As a manager, do you want to engage employees differently in organizational tasks or increase customer loyalty? Pokémon Go showed us not only how gamification can change behavior, but also how a seamlessly mixed reality can blend the actual with the imaginary world. For our real estate example, potential buyers might want to adapt the actual properties of the houses to imagine what the property could look like. They might be interested in what the walls would look like in a different color, or play with its interior design by placing a few pieces of furniture from Sweden throughout the property (Demondern, 2018).
 
 
 Mixed realities combine what is real and possible with what is actual and virtual. The underlying suggestion was that we add possible items (i.e., objects, people) to actual world scenarios. The difference between MR and augmented virtuality might be small, but it is nonetheless important. In contrast to MR, augmented virtuality (AV) refers to computer-generated possible world scenarios augmented with real, virtual constructs (i.e., objects, people). We might create a fictitious world and add actual people or objects into it, much like flight simulators that construct training scenarios that include fictitious elements but also planes of other pilots who train at the same time.
 
 https://search.library.oregonstate.edu/primo-explore/fulldisplay?docid=TN_sciversesciencedirect_elsevierS0097-8493(10)00126-3&context=PC&vid=OSU&lang=en_US&search_scope=everything&adaptor=primo_central_multiple_fe&tab=default_tab&query=any,contains,virtual%20augmented%20reality%20comparison&sortby=rank&offset=0




---------------





