\documentclass[10pt,draftclsnofoot,onecolumn]{IEEEtran}
\usepackage[letterpaper, margin=0.75in]{geometry}

\title{Problem Statement}
\author{
    Kyle Tyler
}

\date{October 9, 2018}

\begin{document}

\maketitle

\begin{center}
    CS 461 - Senior Capstone I \\
    Fall 2018
\end{center}
\bigskip

\begin{abstract}

This project aims to create a virtual reality (VR) program to act as a training application for HP's PageWide Web Press. Currently, training is expensive and can only be done on real presses. By developing virtual reality training environments we can reduce the costs associated with training people to use the presses. Using Unreal engine, the Datasmith CAD model importer, and HP's Microsoft Mixed Reality headset we will build a program to solve the inefficient training problem. Our end goal is to deliver a program that is a complete training package that may even be more effective than using a real press. 

\end{abstract}

\clearpage
\section{Problem}
\bigskip

The problem we are trying to solve is that training for the HP PageWide Web Press product line is currently expensive and hard to do. Our goal is to reduce the cost and impact of training people to use the PageWide Web Press. 
\bigskip 
\par Training currently takes a lot of time and requires access to real presses. These presses are large and hard to come by, so training people to use them with hands on experience is often not easy to do. Because of the cost of the presses it is not currently possible to train people on what to do when something goes wrong, because the risk to the press itself is too great. As of now, people can't really be prepared for things to go wrong until they actually happen. 

\section{Solution}
\bigskip

Our solution is to create one or more virtual reality based environments based on the HP PageWide Web Press equipment and create several interactive learning scenarios to act as training tools. We will use Computer-Aided Design (CAD) models to simulate what using a press would be like in VR. Using VR to train users to use the press would replace real press time, save on training costs, and give a unique experience to trainees. Virtual reality would also allow trainees to see what happens when something goes wrong.  
\bigskip

\par One way that we will ensure that our program is successful is to pull lessons from a place in which our group all has experience: video games. Using the language of video games will allow us to create engaging virtual reality experiences. The field of video games has been using virtual environments to solve problems for years. What we want to do is put a person into a virtual environment and give them the tools they need to solve a problem in the space. By using games as a touchstone we can benefit from techniques we see in games such as: user tutorials, system interaction, and the feeling of immersion. Games must teach players how to use the software itself, and often do so in playful ways. Most games are the result of many complex systems that interact with one another in ways that are fun for a player to interact with. All of this leads to a sense of immersion for the player, and that feeling of immersion can be pushed even further with VR. We hope to use these lessons to build a complex system that is a fun, educational, and immersive experience for users.

\section{Tools and Methods}
\bigskip

The main tools that we will be using for this project are: the Unreal game engine, the Datasmith CAD model importing tool, and the HP Microsoft Mixed Reality headset. Unreal engine is a high production game engine used in many commercial video game projects, as well as for virtual reality experiences and simulation work. It is an incredibly useful tool with good documentation, community support, and also happens to be used by our client company already. Datasmith is a tool developed in part by HP that allows users to import CAD models into their Unreal engine projects more efficiently. This will allow us to take CAD models that HP has and put them directly into our virtual reality application. The last tool, HP's Microsoft Mixed Reality headset, is perhaps the most integral to the project. The fact that our training program is in virtual reality is the main hook of the project. Having a solid piece of hardware that is supported by our client's employer and the Unreal engine will ensure that we can bring our application to the highest possible quality. 

\section{Performance Metrics}
\bigskip

Our project will be completed once we have a functioning virtual reality training program that will allow users to go from having no experience with an HP PageWide Web Press, to being competent enough to use a real press. The trainee may still need to receive some amount of training on a real press, but our goal is to reduce that time as much as possible. Doing this would reduce training costs, free up valuable press time, and allow more in-depth training scenarios then would be possible with a real press.
\bigskip
\par A major metric for success in this project is determining if the training program we develop actually helps people gain the skills and experience they need to use a real press. This means that we will likely need to have people test the program to ensure that it is not only usable, but also engaging and educational. Making something that works will be the easy part, but in order to truly be successful we need to build something that leverages virtual reality and video games to make a positive experience for trainees. The end user for this program may have never used VR before, and have maybe never even played a video game. An important measure of success is that the program is usable for people of varying levels of technical expertise and experience.

\end{document}