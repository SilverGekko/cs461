\textbf{Oct 16, 2018}\\
Progress: We contacted our client and met with him at block 15 to go over initial information about our VR printing project.

Problems: We currently don't all have access to VR headsets in order to get familiar with VR games. Our client told of us that he wanted to approach the development of the VR project as a game development process. As "homework" he told us all to go play VR games to get used to how they work and how tutorials work in VR. We all don't have access to the headsets so we need to get some.

Plans: Get access to a VR headset and play VR games to do our client "homework".\\

\textbf{Oct 21, 2018}\\
Progress:

We met as a team to complete the final draft of the project statement.

Problems:

Out client is currently in Israel and cannot meet with us to discuss the requirements document.

Plans:

We will work together to draft our own set of requirements that the customer will review upon returning from Israel.\\


\textbf{Oct 27, 2018}\\
Progress:

So far our group has been in contact with our sponsor, Tim Holt, about what we should be working on for our group. Tim has said that what we wants us to be doing is playing a few VR games to get ourselves accustomed to the equipment and how VR games are structured vs normal video games. Tim wants to go about making the training application from a game development standpoint and use design philosophies from game design to make a good user experience, specifically taking lessons from tutorials.

Additionally, we also have our outline for our project requirements done and are populating it with our interpretations of what we think the requirements should be.

Problems:

Our project sponsor is still in Israel and cannot devote a lot of time until Monday to helping us with requirements, so much of our requirement document is speculation until our sponsor can get us the training manuals and tell us we are on the right track.

McGrath informed us that the only VR equipment we have access to is in the virtualization lab in Snell hall, which only lets us sign up for hour long blocks. This is fine for us now when all we need to do is play a few games to learn how the tutorials are made and take notes on design. However this poses an issue later in the project if we don't get access to equipment that we can check out for longer periods of time. One hour is not a lot of time to develop games in blocks.

Plans:

After we finish the requirements document we will talk among the group to pick topic that we will write about for our tech reviews since we don't want to accidentally double up on topics.

Also, one of our group members is considering purchasing their own VR headset for personal use. This has the added benefit that we would be able to use it for our group development and not be time restricted. Since two other group member already own VR headsets of some kind, this would speed up development by a lot if we didn't have to use the Snell hall resources to develop.\\

\textbf{Nov 3, 2018}\\
Progress:

Our group was finally able to meet in person with our project sponsor, Tom Holt, at HP to take a tour of the Web Press printers and talk more about expectations of the project. Our group also drafted a requirements document, but it turned into too much of a design document so we created a brand-new document that better fit the "requirements" document description: describing what a solution to the problem statement is required to do, instead of how to do it, which is what our document turned into.

Problems:

Our project sponsor is requiring us to sign an NDA. However, this NDA is not for the entire project, it's just for HP confidential assets such as 3D models and training manuals.

Plans:

Even though we are being required to sign an NDA to use HP assets, it will not get in the way of presenting our project. We will be allowed to create derivative works from the HP-provided CAD files that will not be HP propriety-owned.\\

\textbf{Nov 9, 2018}\\
Progress:

There was not a ton of work for us to do this week. We met in class on Tuesday to peer-review our tech reviews and I got some good feedback about the flow of my paper. I ended up not having to change much from the first draft, since I completed most of it for the first assignment.

Problems:

Our resident game design expect, Stephen, says that we might want to start installing and going through tutorials in the Unreal Engine as it has a steep learning curve. I foresee this being especially challenging for myself as I have very little professional game development experience to my name.

Plans:

Over the remaining weeks of the term I plan on watching Unreal Engine tutorial videos to get up to speed with the rest of my group before the work load starts in Winter term. 

Our team will also work as a group to complete our design document draft, making sure to include the specific user stories that our client request. Last week our client told is that these users stories should include the proper sound ambiance as what is found in the actual R\&D location at HP.\\

\textbf{Nov 16, 2018}\\
Progress:

This week there was not a lot of things for us to do. Last Friday we finished our Tech Reviews and this week in class we talked all about design documents, and got lectured at by McGrath about using spellcheck.

Problems:

Our client has not been very attentive in answering emails. We sent him a progress update for our documentation last Friday and we still have not received word back from him. I also do not feel inclined to come to Thursday lecture if I have to listen to McGrath belittle me as a student for having "poor writing."

Plans:

For the coming holiday weeks we are going to work on the design document to finish it up. We got information from Chris that said each section of the document needs to be 500 words or longer, to ensure that we only write about things that have content.

Also for next term during the design phase we are going to try and set up a weekly meeting with our client so that both us and him feel like we have a good understanding of the project moving forward.\\

\textbf{Nov 22, 2018}
Progress:

We have not made any progress this week on our project.  We did receive an email from our project sponsor Tim about the state of the NDA we have to sign to get access to the source files. We sign it on our own and bring it to the HR person at HP that oversees Tin's department. Again: this NDA only coves the base .stl CAD files that we will be using as a reference, and does not extend to our project in any way. The final project will not be secret and we will be able to call it our own and present it at the career fair.

Problems:

My tech review has not yet been graded so I am unable to review feedback about my writing to apply it to my sections of the design document.

Plans:

This weekend we will finish the design document making sure to follow the ISO standard properly this time. I will ask another one of my group members about their feedback for their tech review in order to better prepare myself for my sections of the design document. \\

\subsubsection{Winter}

\textbf{Jan 11, 2019}\\
Progress:

So far we have contacted McGrath with a request for VR headsets, contacted our client Tim about setting up a weekly meeting with him to go over progress, and set up a weekly meeting with our illustrious TA chris for another progress report.

Problems:

There was some pushback with McGrath when we requested hardware. When we requested two headsets for our group Kevin replied with "I’m sorry, did you really just ask for \$2400 in headsets with a straight face?" This reply did not sit well with our team and were frustrated at first, before McGrath replied again with a bit nicer of a prospect for our team.

Plans:

Wait for Tim to reply and have our first meeting with him to figure out the best place for us to begin developing. Additionally, wait for our headsets from OSU to get in.\\

\textbf{Jan 18, 2019}\\
Progress:

I have installed and done some unreal tutorials with another group member to get myself up to speed. Additionally, our group has received 3 VR headsets to use for development. We have setup a Kanban board for organization are are about to really start our level design.

Problems:

Our project sponsor, Tim, has not yet replied to our email about setting up a weekly meeting. It has been 8 days. I will send a follow up email monday morning in the hopes that he will see it while at work and not let it get bogged down.

Plans:

If Tim still hasn't replied by monday, I will take our group to the webpress to take pictures and measurements to start blocking the levels out in unreal. Aside from that, we will work together on dividing out tasks on the kanban board for completion, as well as brainstorming new tasks.\\

\textbf{Jan 25, 2019}
Progress:

We were able to get our clients response! Tim came to campus today and we're discussed concrete designs for the VR application. we also setup a recurring meeting with him on Fridays.

Problems:

No problems this week.

Plans:

Kyle and Symon checked out the computer graphics lab in batcheller hall for our use of development. each computer has a 1080ti and the unreal engine we we will be using those for development. We have a sample project in a git repo ready to go for additions, so all we need is to start development. we picked Monday afternoons to work for a couple hours together. \\

\textbf{Feb 1, 2019}
Progress:

Ground breaking week for us. We have a working emergency light object blueprint in unreal, a rough 3D model of the press that Stephen made by hand, and a working user interface.

Problems:

When Stephen and I went to HP to meet with Tim to take pictures of the press, the 3 of us got scolded by the floor manager of the web press for not wearing proper safety gear and for not informing them we would be around. Tim in fact did email that person's boss that we would be arriving, the boss just didn't relay that information to the floor manager. This lead to an awkward accusatory conversation where the floor lead said multiple times "we can help you, you just have to help us help you." In the end he was actually really receptive of our mission (even though we neglected to tell him we were taking pictures of his machinery) because it would cut down on the physical training he would have to put his operators through.

Plans:

The floor manager suggested that we could take some of the training classes that HP offers in order to more familiarize ourselves with the press and its operation. Tim is looking into if that is feasible for our timeframe.

Other than that we will continue the work! I continue to add items to the trello board as people tell me what is completed and we get closer to a functional simulator.\\

\textbf{Feb 8, 2019}\\
Progress:

We have fully integrated all of the 3D models of the web press into our VR landscape. We also have added supplementary models to make the space feel more at home, like an actual warehouse aesthetic with rafters and lights, as well as computer desks and rolls of paper.

Problems

Not really a lot of problems this week. We spent a lot of good time developing and made good progress. Our client forgot our meeting today, but it wasn't that critical so we really didn't miss much.

Plans

We drafted up ideas for the VR tutorial we plan to include alongside the training application. These tutorials will teach the user how VR works (looking around, pointing the controllers, teleporting, etc) in an office setting that we plan on adding to the press room. I am going to take a stab at modeling it so I can say I contributed more than just organizational work to this project.\\

\textbf{Feb 15, 2019}\\
Progress:

We have laid the groundwork for all of the specific scenarios we plan to implement by adding the level geometry and 3d models to our main map. There is a new office space to be used for the tutorials.

Problems

not a lot of issues, just scrambling to get ready for alpha release

Plans

implement the first scenario for replacing a printhead from the printbar. once that is done we will have met our clients expectations. we plan on writing the scenario functionality to be easily scalable to be used for other module s since we want HP to be able to turn this into whatever they want.

we are also planning on having a printable PDF that showcases how to use the VR controls for people that have never used VR before. we are going to include the PDF as a poster in the environment so the user can view the controls without removing the headset.\\

\textbf{Feb 22, 2019}\\
Progress:

Tutorials for looking around in VR, teleporting in VR, and interacting with grabable objects are now done. The next step is to complete the unreal blueprint for the training scenarios. Stephen is working on this today and tomorrow and we will finish it on Sunday.

Problems

Not a lot of problems this week. All good development done. We are on track.

Plans

We are getting together Sunday afternoon to implement the final push for alpha release on Monday. We are updating the progress report as we go (currently up to 10 pages) so we don't have to have a scramble at the end of the day tomorrow, which is nice.\\

\textbf{Mar 1, 2019}\\
Progress:

Everyone in our group is finally not sick this week and can work again. Stewart is working on making sure the tutorial works with the HP headset, Kyle and Stephen are working on the training blueprint for scenarios (as well as fixing the bugs I introduced, oops), and Symon and I are thinking about what should go into the voiceover part of the tutorial.

Problems:

Tim was sick this week and didn't get to see what we had done, which is fine for us since we were sick the previous week and also didn't do much besides talk about what needed to be done. However, we did write our progress report draft well (according the the peer reviews I have received), so it wasn't wasted time at all.

Plans:

Finish the last goals of the project such as actually make the training scenario, set up the voiceover and on screen text, polish everything the the warehouse environment, and just generally keep working on the things we hashed out in previous weeks.\\

\textbf{Mar 10, 2019}\\
Progress:

It's actually kind of serendipitous that I forgot about my progress report... Because we finally have a basic scenario done now! Stewart sent me a pull request for it a few hours ago. Its a training scenario blueprint for replacing printheads in the WebPress.

Problems:

Stephen has been unable to work for the past few weeks due to his having the flue. Today he also discovered he is allergic to the antibiotics he was prescribed and had to go to the hospital. Yikes! He seems okay from the messages he sent, but we've been down a developer for the past 2-3 weeks. Additionally, we were not able to meet with Tim this week and he is going on vacation next week so we will probably only be able to show him things after the end of this term. Not to worry though, we went over with him what beta release should look like and he agreed that having one scenario done would be plenty.

Plans:

Polish the hell out of things! We want it to look and act nice for expo. Now that we have a basis of what to work on (the scenario blueprint) we can easily expand it into printheads and e-stop buttons. Anything we want, really.

\textbf{Mar 15, 2019}\\
Progress:

Stewart pushed the final changes to the scenario on Friday. We now have a working training module that shows someone how to remove and replace a printhead from the webpress. When we last talked to our client, this was what he was hoping to accomplish with the project: to be able to demonstrate to upper management that using VR could be an important teaching tool that can save HP time and money.

Problems:

Even though everyone has told me that we are where we need to be in terms of progress on the project, I can't help but shake that feeling of dread as we come closer to the deadline. I really don't want to fail this class and need to take another year of school. I know that happening in extremely rare and most everyone that puts in work will pass. We certainly have put in work, so bare minimum we will pass. The important thing is that we did what we said we would do, and our client likes it.

Plans:

All we need to do for this term is to record the video. Everything could be upgrade once or twice over in terms of visual appeal and style, but I assume that is what we will be doing most of next term to prepare for expo.

\subsubsubsection{Spring}

\textbf{Apr 5, 2019}\\
Progress:

After the class today we met and talked about what final details we needed for code freeze. Most of it boiled down to reorganizing the project folders to be nice. The biggest thing we need to do it wait for a reply from our client to schedule a demo for him to get his feedback about what things we need to change.

Problems:

No problems as of yet.

Plans:

Review the documentation to make sure no revisions are necessary. We don't anticipate needing to do much (or any), but we are still going to check.\\

\textbf{Apr 12, 2019}\\
Progress

We have added the last bits of polish, fixed the controllers (the HP plugin was broken, so we now only rely on the steam API), and started the user guide for code-freeze.

Plans

Continue to try and build the executable and draft the user guide. My task is also to review the requirements and see if anything has been updated.

 
Problems:

Still no reply from Tim yet. Hoping he stops being AWOL in time.\\

\textbf{Apr 19, 2019}\\
Progress:

Nothing new since last week.

Plans:

Continue doing assignments as they come.

Problems:

Still no client verification, but I talked to Kirsten and she told me to submit it anyway because it's out of our hands.\\

\textbf{Apr 27, 2019}\\
Only new thing to report this week is that we got client verification finally, as well as finished our poster. \\

\textbf{May 3, 2019}\\
Nothing new to report this week.\\

\textbf{May 10, 2019}\\
Nothing new to report this week.\\