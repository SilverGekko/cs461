\documentclass[onecolumn, draftclsnofoot,10pt, compsoc]{IEEEtran}
\usepackage{graphicx}
\usepackage{url}
\usepackage{setspace}

\usepackage{geometry}
\geometry{textheight=9.5in, textwidth=7in}

% 1. Fill in these details
\def \CapstoneTeamName{		The HP VR Training Team }
\def \CapstoneTeamNumber{		73}
\def \GroupMemberOne{			Stewart Rodger}
%\def \GroupMemberTwo{			Kyle Tyler}
%\def \GroupMemberThree{			Nicholas Pugliese}
%\def \GroupMemberFour{			Stephen Hoffman}
%\def \GroupMemberFive{			Symon Ramos}
\def \CapstoneProjectName{		Developing Virtual Reality Based Training Experiences for HP's PageWide Web Press Product Line. }
\def \CapstoneSponsorCompany{	Hewlett-Packard, Co}
\def \CapstoneSponsorPerson{		Tim Holt}

% 2. Uncomment the appropriate line below so that the document type works
\def \DocType{		Problem Statement
				%Requirements Document
				%Technology Review
				%Design Document
				%Progress Report
				}
			
\newcommand{\NameSigPair}[1]{\par
\makebox[2.75in][r]{#1} \hfil 	\makebox[3.25in]{\makebox[2.25in]{\hrulefill} \hfill		\makebox[.75in]{\hrulefill}}
\par\vspace{-12pt} \textit{\tiny\noindent
\makebox[2.75in]{} \hfil		\makebox[3.25in]{\makebox[2.25in][r]{Signature} \hfill	\makebox[.75in][r]{Date}}}}
% 3. If the document is not to be signed, uncomment the RENEWcommand below
\renewcommand{\NameSigPair}[1]{#1}

%%%%%%%%%%%%%%%%%%%%%%%%%%%%%%%%%%%%%%%
\begin{document}
\begin{titlepage}
    \pagenumbering{gobble}
    \begin{singlespace}
    	% \includegraphics[height=4cm]{coe_v_spot1}
        \hfill 
        % 4. If you have a logo, use this includegraphics command to put it on the coversheet.
        %\includegraphics[height=4cm]{CompanyLogo}   
        \par\vspace{.2in}
        \centering
        \scshape{
            \huge CS 461 Fall Term Capstone \DocType Rough Draft \par
            {\large\today}\par
            \vspace{.5in}
            \textbf{\Huge\CapstoneProjectName}\par
            \vfill
            {\large Prepared for}\par
            \Huge \CapstoneSponsorCompany\par
            \vspace{5pt}
            {\Large\NameSigPair{\CapstoneSponsorPerson}\par}
            {\large Prepared by }\par
            Group\CapstoneTeamNumber\par
            % 5. comment out the line below this one if you do not wish to name your team
            \CapstoneTeamName\par 
            \vspace{5pt}
            {\Large
                %\NameSigPair{\GroupMemberOne}\par
                %\NameSigPair{\GroupMemberTwo}\par
                %\NameSigPair{\GroupMemberThree}\par
                %\NameSigPair{\GroupMemberFour}\par
                %\NameSigPair{\GroupMemberFive}\par
            }
            \vspace{20pt}
        }
        \begin{abstract}
        % 6. Fill in your abstract    
        	This document covers Hewlett-Packard Company's problems with their current PageWide Web Press training program, specifically the time and money that HP believes is being needlessly spent on in-person training sessions for the press. This is detailed in section 1, "Problem Description". In section 2, "Proposed Solution", the outlined solution to the problem is to move the training to a virtual reality space using the Unreal Engine, Microsoft's AR headset which HP helped to develop, and other tools, so that trainees can access the press and instructional materials at a much lower cost without any loss of information retention. In section 3, "Performance Metrics", the specific goals of the project are outlined. In short, this solution will produce a piece of VR software that covers the basics of the press and can function as a success/failure training tool. 
        \end{abstract}     
    \end{singlespace}
\end{titlepage}
\newpage
\pagenumbering{arabic}
\tableofcontents
% 7. uncomment this (if applicable). Consider adding a page break.
%\listoffigures
%\listoftables
\clearpage

% 8. now you write!
\section{Problem Description}

HP's PageWide Web Press is a very large, very expensive, piece of machinery. The supply of which is also fairly limited. As a result it can be difficult to get people trained on the machines. As the room-sized press can not be shipped out to different locations on demand, many trainees will need to be flown out to the location where a training machine setup is due to this lack of availability; this drives up the cost of training as HP must now pay for the airfare and a form of compensation for travel expenses. This also takes up valuable time as it complicates training scheduling. In addition, the number of training setups can be inadequate for the number of people who need to be trained. These issues, compounded with the fact that HP can not break a press just to show trainees what happens if you do it wrong, creates an less than ideal environment for training. HP is looking for a way to get more people trained on their PageWide Web Press', faster, and at a reduced cost.     

\section{Proposed Solution}

To attempt to solve these issues, we will be developing a virtual reality training equivalent for the PageWide Web Press. By recreating the PageWide Web Press Product Line in a interactive virtual environment, a person can be trained in the press' various functions and flaws, observe failure, and become familiarized with the controls, all with a VR headset and compatible computer setup. As these things either already exist or are easily shipped and set up at every HP location, which will dramatically reduce training cost and time as nobody would need to be flown to a training location and set up with lodgings. All that needs to be done it to apply the methods, tools, and tricks used by the game development industry for decades (and about half a decade for virtual reality headsets), to create a environment and experience that is user friendly enough to be used by trainees, and detailed enough to actually function as an effective training tool. This solution would involve the use of the Unreal Game engine, Microsoft's AR Headset, Datasmith (an Unreal Engine utility tool), and CAD (Computer-aided design) Models.  

\section{Performance Metrics}

The project will reach a completed state when:

1. A VR environment is construction with a PageWide Web Press in the scene. 

2. The user can move around within the scene using the provided VR headset, and the user can interact with the press using the provided controller. 

3. The interactive parts of the VR press simulate the same major functions in virtual reality that would be covered in the physical reality training.   

4. The VR press simulates failure when the VR press is operated incorrectly (control's used in the wrong order) in at least one (ideally all) of the functions in the training.

5. Outside individuals with no prior knowledge can complete the training successfully, without triggering a failure (when allowed repeat attempts), using only the instructions and prompts within the VR simulation. 

\end{document}